Należy zwrócić uwagę, iż każda poniższa strona poświęcona jest innemu obiektowi. W sprawnej nawigacji po tym rozdziale może pomóc spis rysunków dostępny na końcu pracy.
\foreach \y in {1,...,21}{
	\begin{figure}[H]
		\centering
		\includegraphics[width=140mm]{G1-tf-\y a}
		\caption{Obiekt G1-tf\y a}
		\label{fig:G1-tf-\y a}
	\end{figure}
	\begin{figure}[H]
		\centering
		\includegraphics[width=140mm]{G1-tf-\y b}
		\caption{Obiekt G1-tf\y b}
		\label{fig:G1-tf-\y b}
	\end{figure}
}

\foreach \y in {1,...,21}{
	\begin{figure}[H]
		\centering
		\includegraphics[width=140mm]{G2-tf-\y a}
		\caption{Obiekt G2-tf\y a}
		\label{fig:G2-tf-\y a}
	\end{figure}
	\begin{figure}[H]
		\centering
		\includegraphics[width=140mm]{G2-tf-\y b}
		\caption{Obiekt G2-tf\y b}
		\label{fig:G2-tf-\y b}
	\end{figure}	
}

\foreach \y in {1,...,10}{
	\begin{figure}[H]
		\centering
		\includegraphics[width=140mm]{G3-tf-\y a}
		\caption{Obiekt G3-tf\y a}
		\label{fig:G3-tf-\y a}
	\end{figure}
	\begin{figure}[H]
		\centering
		\includegraphics[width=140mm]{G3-tf-\y b}
		\caption{Obiekt G3-tf\y b}
		\label{fig:G3-tf-\y b}
	\end{figure}
}

\foreach \y in {1,...,6}{
	\begin{figure}[H]
		\centering
		\includegraphics[width=140mm]{G4-tf-\y a}
		\caption{Obiekt G4-tf\y a}
		\label{fig:G4-tf-\y a}
	\end{figure}
	\begin{figure}[H]
		\centering
		\includegraphics[width=140mm]{G4-tf-\y b}
		\caption{Obiekt G4-tf\y b}
		\label{fig:G4-tf-\y b}
	\end{figure}
}
\foreach \y in {1,...,9}{
	\begin{figure}[H]
		\centering
		\includegraphics[width=140mm]{G5-tf-\y a}
		\caption{Obiekt G5-tf\y a}
		\label{fig:G5-tf-\y a}
	\end{figure}
	\begin{figure}[H]
		\centering
		\includegraphics[width=140mm]{G5-tf-\y b}
		\caption{Obiekt G5-tf\y b}
		\label{fig:G5-tf-\y b}
	\end{figure}
}
\foreach \y in {1,...,9}{
	\begin{figure}[H]
		\centering
		\includegraphics[width=140mm]{G6-tf-\y a}
		\caption{Obiekt G6-tf\y a}
		\label{fig:G6-tf-\y a}
	\end{figure}
	\begin{figure}[H]
		\centering
		\includegraphics[width=140mm]{G6-tf-\y b}
		\caption{Obiekt G6-tf\y b}
		\label{fig:G6-tf-\y b}
	\end{figure}
}
\foreach \y in {1,...,36}{
	\begin{figure}[H]
		\centering
		\includegraphics[width=140mm]{G7-tf-\y a}
		\caption{Obiekt G7-tf\y a}
		\label{fig:G7-tf-\y a}
	\end{figure}
	\begin{figure}[H]
		\centering
		\includegraphics[width=140mm]{G7-tf-\y b}
		\caption{Obiekt G7-tf\y b}
		\label{fig:G7-tf-\y b}
	\end{figure}
}
\foreach \y in {1,...,11}{
	\begin{figure}[H]
		\centering
		\includegraphics[width=140mm]{G8-tf-\y a}
		\caption{Obiekt G8-tf\y a}
		\label{fig:G8-tf-\y a}
	\end{figure}
	\begin{figure}[H]
		\centering
		\includegraphics[width=140mm]{G8-tf-\y b}
		\caption{Obiekt G8-tf\y b}
		\label{fig:G8-tf-\y b}
	\end{figure}
}
\foreach \y in {1,...,10}{
	\begin{figure}[H]
		\centering
		\includegraphics[width=140mm]{G9-tf-\y a}
		\caption{Obiekt G9-tf\y a}
		\label{fig:G9-tf-\y a}
	\end{figure}
	\begin{figure}[H]
		\centering
		\includegraphics[width=140mm]{G9-tf-\y b}
		\caption{Obiekt G9-tf\y b}
		\label{fig:G9-tf-\y b}
	\end{figure}
}
